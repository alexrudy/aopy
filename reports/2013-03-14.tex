\documentclass[12pt]{article}
%--- PACKAGES ---
%\usepackage{fontspec} %XeTeX Fonts
\usepackage[usenames,dvipsnames]{color}

% Paper Geometry
\usepackage{geometry} 
\geometry{letterpaper} %Paper Size
\usepackage{fancyhdr}
\usepackage{lscape}
%\usepackage{fullpage}
%\usepackage{lastpage}

% Graphics Packages
\usepackage[xetex]{graphicx}
%\usepackage{epstopdf}
\usepackage{wrapfig}
\usepackage{float}
\usepackage{rotating}


% Problem Statement Commands
\usepackage{boxedminipage}
\usepackage{enumitem}

% Mathematics and Symbols
\usepackage{amsmath} 
\usepackage{amssymb}

% Text and References
\usepackage{varioref}
\usepackage{url}
\usepackage{natbib}
\usepackage{setspace}
\usepackage{xltxtra}


%--- GRAPHICS COMMANDS ---%
\DeclareGraphicsRule{.tif}{png}{.png}{`convert #1 `dirname #1`/`basename #1 .tif`.png}
%\renewcommand{\labelenumi}{\textbf{(\alph{enumi})}}

%---- XeLaTeX COMMANDS ---%
%\setromanfont{Palatino}

%---- META-DATA INFORMATION ----%
% Set up for Homework from courses: change titlepages etc.
\newcommand{\gotitle}{\today\ Report on Wind Prediction Methods}
\newcommand{\goauthor}{Alexander Rudy}

%---- TITLE INFORMATION ----%
% LaTeX Metadata commands required for \maketitle, which is actually pretty bad
\title{\gotitle}
\author{\goauthor}
\date{\today}                                                         

%--- HEADER INFORMATION ----         
\pagestyle{fancy}

\fancyhf{}
%Header Content
\lhead{\goauthor}
\chead{\gotitle}
\rhead{}
%Footer Content
\cfoot{ - \thepage \ - }
\lfoot{}
\rfoot{}
% Header
\headheight 40 pt
\headsep 10pt
\renewcommand{\headrulewidth}{0 pt}
\renewcommand{\footrulewidth}{0 pt}

% Meta Rules
\newcommand{\note}[1]{{\color{blue}\emph{#1}}}
\newcommand{\scrap}[1]{{\color{green}#1}}
\newcommand{\error}[1]{{\color{red}#1}}

%--- CUSTOM COMMANDS ---
\renewcommand{\ll}{\left} %Left Parenthesis
\newcommand{\rr}{\right} %Right Parenthesis
\newcommand{\txt}{\textrm} %text mode shortcut
\renewcommand{\v}[1]{\vec{#1}} %Vector Notation
\newcommand{\unit}[1]{\txt{ #1}} %Units Mode
\renewcommand{\b}[1]{\mathbf{#1}} %Bold in Math Mode
\newcommand{\E}[1]{\times 10^{#1}} %Times 10 to the [argument]
\renewcommand{\deg}{^{\circ}} %degrees symbol

%--- Some MKS Numbers for Astro ---%
\newcommand{\g}{9.8\ \txt{ m} \txt{ s}^{-2}} %Insert g
\renewcommand{\G}{6.7 \times 10^{-11} \txt{ N m}^2\txt{ s}^{-2}} %Insert G
\newcommand{\ms}[1]{\txt{ m} \txt{ s}^{#1}} %Meters per second^[argument]

% General Formatting Rules
\newcommand{\HRule}{\rule{\linewidth}{0.5mm}} %Horizontal Line
\newcommand{\pref}[1]{(\ref{#1})} %Parenthesis References

% QM Commands
\newcommand{\bra}{\langle}
\newcommand{\ket}{\rangle}

%--- DOCUMENT ---
\begin{document}

\begin{figure}[htbp]
   \centering
   \includegraphics[width=\textwidth]{../figures/histogram_FT_keck_simulated_sim_0} % requires the graphicx package
   \caption{This is the wind map for an artificially generated wind analyzed by the `Fourier Transform' method. It shows the artificially inserted wind at $-25 \unit{m/s}$ as expected. The triangle shows the peak value, and the circle shows the measured image Center-of-Mass. Neither is a sophisticated peak finding method.}
   \label{histogram_FT_keck_simulated_sim_0}
\end{figure}

\begin{figure}[htbp]
   \centering
   \includegraphics[width=\textwidth]{../figures/contours_all_keck_simulated_sim_0.png} % requires the graphicx package
   \caption{This is the wind map for an artificially generated wind. It shows the artificially inserted wind at $-25 \unit{m/s}$. The triangle shows the peak value, and the circle shows the measured image Center-of-Mass. The individual methods are GN: Gauss-Newton Minimization, 2D: 2D Binary Search, XY: Split Binary Search, and FT: Fourier Transform}
   \label{contours_all_keck_simulated_sim_0}
\end{figure}


\begin{figure}[htbp]
   \centering
   \includegraphics[width=\textwidth]{../figures/timeseries_GN_keck_simulated_sim_0.png} % requires the graphicx package
   \caption{This is a timeseires plot of the measured wind with the Gauss-Newton Method, smoothed for a temporal updating response of $100\unit{Hz}$. The blue lines show results from my runs of Luke's code, the black X's show Luke's code's direct output. The good agreement is a nice sanity check. The smoothing is a flat window filter. Notice that the Gauss-Newton wind never seems to predict the full $25 \unit{m/s}$ wind expected in the $x$ direction.}
   \label{timeseries_GN_keck_simulated_sim_0}
\end{figure}

\begin{figure}[htbp]
   \centering
   \includegraphics[width=\textwidth]{../figures/histogram_FT_altair_20070417.png} % requires the graphicx package
   \caption{This is the wind map for an altair measured wind. It shows what appears to be a single strong layer at $\left(22,10\right)\unit{m/s}$.}
   \label{histogram_FT_altair_20070417}
\end{figure}

\begin{figure}[htbp]
   \centering
   \includegraphics[width=\textwidth]{../figures/contours_all_altair_20070417.png} % requires the graphicx package
   \caption{This is the wind map for an altair measured wind. As expected, Luke's methods form a center-of-mass measurement in between markers of Lisa's layers.}
   \label{contours_all_altair_20070417}
\end{figure}
\end{document}
